% document settings
\documentclass[11pt,a4paper]{scrartcl}		% article
\usepackage[utf8x]{inputenc}					% input UTF-8
\usepackage{geometry}						% for layout
\usepackage[ngerman]{babel}					% use english or ngerman
\usepackage{microtype}						% better font kerning
\usepackage[hidelinks]{hyperref}			% use invisible clikable links

% layout
\geometry{hmargin=2.5cm, top=3cm, bottom=2.5cm}	% borders
\pagestyle{empty}							% plain/empty (w/wo pagenumbers)
\setlength{\parindent}{0pt}					% indent at new line

% fonts
\renewcommand{\rmdefault}{ptm}				% cmr, ptm, ppl, pnc, pbk
\renewcommand{\sfdefault}{cmss}				% cmss, phv pag
\renewcommand{\familydefault}{\rmdefault}	% set main font

\title{Schafkopfn}

\makeatletter
\def\maketitle{
	\begin{center}
		\textbf{\huge \@title}
	\end{center}
}
\makeatother



% document begin
\begin{document}
	% initial vertical skip

	\maketitle



	Gspuit wern derf: Sau Spui, Farb Solo, Wenz

	Zoit werd: Sau Spui zenerl, Solo zwanzgerl, jeder Lafade zenerl, Schneider zenerl, Schwarz verdoppelt



	%\section{Verostoitn}
	
	Wer an Schafkopf Abend verostoitn wui, der braucht
	\begin{itemize}
		\item A boarisches Bladdl mit zwoaradreisg Kartn
		\item Gnua koits Bier füa via Leit
		\item An Disch mit via Stui (und a Liacht wenns späda werd)
	\end{itemize}


	\section{Goid}

	\begin{tabular}{lll}
		Sau Spui & zenerl \\
		Solo & zwanzgerl \\
		Schneider & zenerl \\
		Schwarz & doppelt \\ 
	\end{tabular}	


	\section{Spuiarten}
	Wenn oana sagt er mog „spuin“ dann guid des. Außer Du wuist a Solo spuin, dann sagst du „spuist aa“ und Du derfst spuin. Statt am Solo derfst aa an Wenz spuin. Wenn oana sagt er „spuit aa“, dann kimmst nua no mim „Solo Du“ drüber. Und übern Solo Du kimmst nua mim „Solo Sie“ aber dann hoasd an Nagl in der Dia.


		\subsection{Sau Spui}
		Beim Sau Spui san die Ober, die Unter und olle Heazn Drimpf.

		Wennst a Sau Spui spuist, musst oane Sau osang mit derst dann zam spuist. Du kannst mit der „Oiden“, der „Blauen“ oder der „Hundsgfickten“ spuin.
		Dafür brauchst oane Foabkartn von der gleichn Foab wie die Sau diest osang möchst. Wennst koane Foabkartn hast um a Sau zu rufn, dann bist gsperrt und kannst leida koa Sau Spui spuin.


			\subsubsection{Schmiern}
			Die gsuachte Sau derf erst abm vorletztn Stich gschmiert wern.

			\subsubsection{Davo Lafa}
			Die gsuachte Sau derf davo lafa, solang di gsuachte Farb no ned o'gspuid wordn is. Davo lafa derfst wennst mindestenst via Kartn von der gsuachten Farb auf da Hand hast und dro bist. Davo lafa hoasd also du kimmst ausi und spuist \emph{ned} die gsuachte Sau sondern a andre Kartn von der gsuachten Farb. Hast mi?

		
		\subsection{Solo}
		Wenn'st moanst die heilige Maria hat dei Kartn küsst, dann kannst aa a Solo spuin. Des bedeit, deine Freinde san jetzt deine Feinde und Du spuist ganz aloa gega drei. Dafüa kannst aa an ordnentlichn Batzn Goid abstaub'm.

		Als Trost derfst de Farb osagn, die Trumpf is. 		

			\subsubsection{Wenz}
			Der Wenz is a Solo, wo nua die via Unter Drimpf san.

			\subsubsection{Tout und Sie}
			Das Solo Tout („Du“) is a Solo des Du nur gwinnst wenn Du alle Stiche machst. Der Solo Oansatz wird verdoppelt.
			Beim Solo Sie muss der Spieler olle Ober und Unter auf der Hand ham. A Sie werd ned gspuit sonder da wern oanfach olle Kartn aufn Disch gnallt.

			\subsubsection{Schiaba Rundn}		
			Wenn'st a normals Solo (koa Wenz) verlorn hast, dann werd ned lang gweint, sondern a Schiaba Rundn g'spuit. In der Schiaba Rundn wern die beiden hechsten Drimpf (Der Oide und da Blaue) in die Mittn g'legt und der erste Spieler derf die Kartn nehmen und zwoa andere (Spotzen) weida schiam. Der zwoate Spieler schiabt aa zwoa Kartn weida und derf zwoa Tipps geben. („Di Farb isses“ oder „Di Farb isses ned“). 
			Wenn oana moant er is der Kenig in dem Spui dann derf er zwoa moi aufn Tisch klopfn. Mehr ned.



		\subsection{Stock}
		Wenn Jeda zu feig zum spuin is, dann werd der Oansatz von am Sau Spui in den Stock zoit. 


		\subsection{Koa Spui}
		An Geier, Farb-Wenz, oder Ramsch spuin nur die Breissn. Und merk Dia glei oans: Heiratn dern aa nua die Deppn. Also vergiss des Ois, des hat nix mim Schafkopfn zum do.  %dern?



	\section{Wias Spui oblaft}

	Der Geber mischt die Kartn und lässt den Spieler rechts von erm obhem. Es derf dabei ned mera wia dreimoi obghom wern. Durchs obhem is des vorige Spui samt olle Fehler Schnee von Gestern. 

	Danoach teilt der Geber jedem Spieler der Reih nach via Kartn aus bis olle Kartn ausgem san. Der Spieler links vom Geber is dabei der Erste. 

	Dann sagt jeder Spieler in der glaichn Reinfolge o ob er spuin mog oder ned. Wer ned spuit sogt „weida“, wer spuin mog sagt „spuin“. Wenn no oarna spuin mog, muss er mindestens a Solo spuin und sagt „aa spuin“. Dann kimts drauf o wers höhere Spui hod. Dabei is a Solo (oda Wenz) mera Wert wie a Sau Spui und a Solo Du mera wie a normals Solo. Wenn beide a Solo spuin woin dann derf der der z'erst gfragt war spuin.  

	Soboid fest staat, wer spuit sogt derjenige des Spui gnau o, also aa welche Foab.

	Dann gehts richtig los und der Erste haut die erste Kartn aufn Disch.


	Während dem Spui derf a Jeda den letzten Stich nochmoi oschaun.
	A Stich guid als obglegt, sobald die Kartn umdraht worn san.



	\section{Wias daira werd}

	\begin{tabular}{ll|ll}
		Sau & 11 Augn & Zehna & 10 Augn\\
		Kenig & 4 Augn &  Ober & 3 Augn\\
		Unter & 2 Augn & 7, 8, 9 & koane Augn
	\end{tabular}

	


		\subsection{Schneider}
		%Wenn die wo spuin, mit oananeinzg oda mera Augn gwinnen, dan san die andern Schneider. 
		Wenn die wo ned spuin, mit weniger als dreißg Augn verliern, sans Schneider. Ansonsten sans Schneider frei.
		Füa die wo spuin guid des gleiche, aba mit oanadreißg Augn.


		%Wenn die andern dreisig oder mera Augn ham, sans Schneider frei. 



		\subsection{Schwarz}
		Wenn oane Partei gar koane Augn zam bracht hat, dann sans schwarz und der Oansatz wird verdoppelt.


		\subsection{Stoß}
		In jedem Spui ko der, der ned spuit an Stoß gebn. Stoßn derfst nua beim ausspuin von deiner erstn Kartn.
		Wen Oana an Stoß gibt werd der Oansatz verdoppelt und derjenige übernimmts Spui.
		Des bedeit, der Stoßende braucht oanasechzg Augn ums Spui zu gwinna. Des hod scho manchen die Freindschaft kost, also überleg Dir vorher wast sogst.

		Beim Stoß kann bloß oamoi retour gebn wern.


		\subsection{Die Lafanden}
		Wenn oane Partei merare Drimpf ab dem Hechsten (dem Oiden) in Folge hod, dann hams Lafade.
		Beim Sau Spui und beim Solo zoin die Lafanden ab drei, beim Wenz ab zwoa.
		Jeda Lafade zoit so fui wia da Oansatz von am Sau Spui. 




	\section{Wiama gscheid spuit}


		Am besten is, wenn dei 

	Wennst a Solo mit Spotzen spuist, dann solltest schaun dass Du die ned zspat wegspuist. Wennst des ned derst, kanns passiern, dass da di dairen Stich von de Andern hintnaus no einibräsln.



\subsection{Der Oide gibt die letzte Rundn}
Wenn'st Hoam zua deiner Oiden wuist und nimma mit o schaun kannst, wia Dir deine Spetzl des Goid aus der Taschn ziehn, dann kannst verlanga das der Oide die letzte Rundn gibt. Aba Obacht: Wenn deine Spetzl arg gmain san, dann druckens Dir noch a Bock-Rundn aufs Aug. In der Bock-Rundn zoit ois doppelt so vui.




\end{document}
