% document settings
\documentclass[11pt,a4paper]{scrartcl}		% article
\usepackage[utf8x]{inputenc}					% input UTF-8
\usepackage{geometry}						% for layout
\usepackage[ngerman]{babel}					% use english or ngerman
\usepackage{microtype}						% better font kerning
\usepackage[hidelinks]{hyperref}			% use invisible clikable links

% layout
\geometry{hmargin=2.5cm, top=3cm, bottom=2.5cm}	% borders
\pagestyle{empty}							% plain/empty (w/wo pagenumbers)
\setlength{\parindent}{0pt}					% indent at new line

% fonts
\renewcommand{\rmdefault}{ptm}				% cmr, ptm, ppl, pnc, pbk
\renewcommand{\sfdefault}{cmss}				% cmss, phv pag
\renewcommand{\familydefault}{\rmdefault}	% set main font

\title{Schafkopfn}

\makeatletter
\def\maketitle{
	\begin{center}
		\textbf{\huge \@title}
	\end{center}
}
\makeatother



% document begin
\begin{document}
	% initial vertical skip

	\maketitle


	%\section{Verostoitn}
	
	Wer an Schafkopf Abend verostoitn wui, der braucht
	\begin{itemize}
		\item A boarisches Bladdl mit zwoaradreisg Kartn
		\item Genug gekühltes Bier füa via Leit
		\item An Disch mit via Stui (und a Liacht wenns späda werd)
	\end{itemize}


	\section{Goid}



	\section{Spuiarten}
	Wenn oana sagt er mog „spuin“ dann guid des. Außer Du wuist a Solo spuin, dann sagst du „spuist aa“ und Du derfst spuin. Statt am Solo derfst aa an Wenz spuin. Wenn oana sagt er „spuit aa“, dann kimmst nua no mim „Solo Du“ drüber. Und übern Solo Du kimmst nua mim „Solo Sie“ aber dann hoasd an Nagl in der Dia.


		\subsection{Sau Spui}
		Wenn Du spuist und koana spuit aa, dann musst Du die Sau o'sagn mit der'st spuist.
		Du kannst mit der „Oiden“, der „Blauen“ oder der „Hundsgfickten“ spuin.


			\subsubsection{Davo Lafa}
			Die gsuachte Sau derf davo lafa, solang di gsuachte Farb no ned o'gspuid wordn is. Davo lafa derfst wennst mindestenst via Kartn von der gsuachten Farb auf da Hand hast und dro bist. Davo lafa hoasd also du kimmst ausi und spuist \emph{ned} die gsuachte Sau sondern a andre Kartn von der gsuachten Farb. Hast mi?

		
		\subsection{Solo}
		Wenn'st moanst die heilige Maria hat dei Kartn küsst, dann kannst aa a Solo spuin. Des hoasd, deine Freinde san jetzt deine Feinde und Du spuist ganz aloa. Dafüa kannst aa an ordnentlichn Batzn Goid abstaub'm.


			\subsubsection{Schiaba Rundn}		
			Wenn'st a Solo verlorn hast, dann werd ned lang gweint, sondern a Schiaba Rundn g'spuit. In der Schiaba Rundn wern die beiden hechsten Drimpf (Der Oide und da Blaue) in die Mittn g'legt und der erste Spieler derf die Kartn nehmen und zwoa andere (Spotzen) weida schiam. Der zwoate Spieler schiabt aa zwoa Kartn weida und derf zwoa Tipps geben. („Di Farb isses“ oder „Di Farb isses ned“). 
			Wenn oana moant er is der Kenig in dem Spui dann derf er zwoa moi aufn Tisch klopfn. Mehr ned.


		\subsection{Stock}
		Wenn Jeda zu feig zum spuin is, dann wird der Oansatz von am Sau Spui in den Stock zoit. 


		\subsection{Koa Spui}
		An Geier, Farb-Wenz, oder Ramsch spuin nur die Preisn. Und merk Dia glei oans: Heiratn dern nua die Deppn. Also vergiss des Ois, des hat nix mim Schafkopfn zum do.  %dern?



	\section{Wias daira werd}

		\subsection{Schneida}


		\subsection{Schwoarz}


		\subsection{Stoß}
		In jedem Spui ko der, der ned spuit an Stoß gebn. Stoßn derfst nua beim ausspuin von deiner erstn Kartn.
		Wen Oana an Stoß gibt werd der Oansatz verdoppelt.
	Wer an Stoß gibt, übernimmts spui. Des hoast, der Stoßende braucht oanasechzg Augn ums Spui zu gwinna. Des hod scho manchen die Freindschaft kost, also überleg Dir vorher wast sagst.


		\subsection{Die Lafanden}
		Jeda Lafade zählt so fui wia da Oansatz von am Sau Spui. 

		Beim Solo zähln die Lafanden ab drei, beim Wenz ab zwoa.







\subsection{Der Oide gibt die letzte Rundn}
Wenn'st Hoam zua deiner Oiden wuist und nimma mit o schaun kannst, wia Dir deine Freinde des Goid aus der Taschn ziehn, dann kannst verlanga das der Oide die letzte Rundn gibt. 




\end{document}
